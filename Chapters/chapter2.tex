\chapter{无穷复数项级数}
\begin{introduction}
    \item 复数项级数
    \item 复幂级数
    \item Taylor级数
    \item Laurent级数
    \item 解析延拓
\end{introduction}

\section{复数项级数}
    \subsection{复数项级数}
    复数项级数和实数项级数完全类似,直接快进到审敛法(

    \subsection{复数项级数的审敛定理}
    这里给出复数项级数的审敛定理.
    \begin{enumerate}
        \item 比值判别法\\
            若$\exists N \subset \mathbbm{N}$,对$\forall n > N$,都有
            $|u_n| < v_n$,并且$\sum_{n = 0}^{\infty}v_n$收敛,则$\sum_{n=0}^{\infty}|u_n|$收敛,即$\sum_{n=0}^{\infty}u_n$绝对收敛.
            若$u_n>v_n>0$,且$\sum_{n=0}^{\infty}v_n$发散,则$\sum_{n=0}^{\infty}|u_n|$发散.
        \item 比值判别法\\
            若存在与$n$无关的常数$\rho$,使得$\left|\dfrac{u_{n+1}}{u_n}\right|<\rho<1$,则级数$\sum_{n=0}^{\infty}u_n$绝对收敛;
            若$\left|\dfrac{u_{n+1}}{u_n}\right|>\rho>1$,则级数发散.
        \item $d'Alembert$判别法\\
            若$\varlimsup_{n\to\infty}|\dfrac{u_{n+1}}{u_n}|<1$,则级数$\sum_{n=0}^{\infty}u_n$绝对收敛;
            若$\varliminf_{n\to\infty}|\dfrac{u_{n+1}}{u_n}|>1$,则级数发散.
        \item $Cauchy$判别法\\
            同$d'Alembert$判别法,只需将$\varlimsup_{n\to\infty}|\dfrac{u_{n+1}}{u_n}|$和$\varliminf_{n\to\infty}|\dfrac{u_{n+1}}{u_n}|$
            均更换为上极限$\varlimsup{{|u_n|}^{1/n}}$.\\
            若$d'Alembert$判别法和$Cauchy$判别法中的两式均为$1$,我们可以使用下面的$Gauss$判别法.
        \item $Gauss$判别法\\
            若$\lim_{n\to\infty}|\dfrac{u_{n+1}}{u_n}|=1$,我们将邻项比值写成如下形式:
            \begin{align*}
                \frac{u_n}{u_{n+1}}=1+\frac{\mu}{n}+O(n^{-\lambda})
            \end{align*}
            这里我们一般通过$Taylor$展开得到.其中$\mu=a+ib,\ \lambda > 1$,若$a>1$,则级数绝对收敛,反之发散.
    \end{enumerate}
    \begin{example}
        判断级数$\sum_{n=1}^{\infty}\dfrac{i^n}{n^\alpha}\,(\alpha>0)$的收敛性和绝对收敛性.
    \end{example}
    \begin{solution}
        我们可以看到,$\sum_{n=1}^{\infty}|\dfrac{i^n}{n^\alpha}|=\sum_{n=1}^{\infty}\dfrac{1}{n^\alpha}$,
        这就是实数项级数中的$p$级数,所以当$\alpha\leq 1$时级数发散,反之绝对收敛.这里我们使用$Gauss$判别法得到相同结论:
        \begin{align*}
            \frac{c_n}{c_{n+1}}=\frac{1/n^\alpha}{1/(n+1)^\alpha}=\left( 1+\frac{1}{n} \right)^\alpha=1+\frac{\alpha}{n}+O\left(\frac{1}{n^2}\right)
        \end{align*}
        当$\alpha \leq 1$时级数发散,反之收敛.
    \end{solution}

\section{复幂级数}
    \subsection{复幂级数的定义和性质}
        \begin{definition}[复幂级数]\label{def:complex_power_series}
            我们将形如$\sum_{k=0}^{\infty}(z-z_0)^k$的复函数项级数称为复幂级数.
        \end{definition}

        \begin{proposition}
            若幂级数$\sum_{k=0}^{\infty}a_k(z - z_0)^k$在点$z_1=\neq z_0$处收敛,则该级数在$|z-z_0|<|z_1-z_0|$处均收敛;
            同样地,若幂级数$\sum_{k=0}^{\infty}b_k(z - z_0)^{(-k)}$在点$z_2=\neq z_0$处收敛,则该级数在$|z-z_0|>|z_2-z_0|$处均收敛
        \end{proposition}

    \subsection{复幂级数的收敛半径计算}
        这部分和实幂级数的计算方法完全一样,但是在计算时要时刻提醒自己算的是复数.
        \begin{enumerate}
            \item 由$Cauchy$判别法,我们可以得到$Cauchy-Hadamard$公式:
                \begin{align}
                    R=\varliminf_{n\to\infty}\left|\frac{1}{c_n}\right|^{1/n}
                \end{align}
            \item 由$d'Alembert$判别法,我们可以得到:
                \begin{align}
                    R=\lim_{n\to\infty}\left|\frac{c_n}{c_{n+1}}\right|
                \end{align}
        \end{enumerate}
        \begin{remark}
            $Cauchy-Hdamard$公式在任意条件下均成立,但$d'Alembert$公式要求有$\lim_{n\to\infty}|c_n/c_{n+1}|$存在.
            求收敛半径时一般先用后者,行不通再使用前者.使用前者时一般要用到重要极限,请牢牢记住$\lim_{f(x)\to\infty}\left(1+\dfrac{1}{f(x)}\right)^{f(x)}=e$.
        \end{remark}

\section{Taylor级数}
    \subsection{Taylor级数的定义}
        \begin{definition}[Taylor级数]\label{def:taylor_series}
            设$f(z)$在圆域$C$内解析,$z_0$为$C$内任一点,则函数$f(z)$可以展开为幂级数形式.
            \begin{align}
                f(z)=f(z_0)+f'(z_0)(z-z_0)+\cdots
                =\sum_{n=1}^{\infty}\dfrac{f^{(n)}(z_0)}{n!}(z-z_0)^n
            \end{align}
        \end{definition}
        \begin{definition}[Maclaurin级数]\label{def:maclaurin_series}
            这里如果取$z_0=0$,我们可以得到$Maclaurin$级数.
            \begin{align*}
                f(z)=f(0)+f'(0)z+\cdots=\sum_{n=0}^{\infty}\frac{f^{(n)}(0)}{n!}z^n
            \end{align*}
        \end{definition}

    \subsection{重要的Taylor级数}
        这部分请务必牢记,这些$Taylor$级数(都是在$z_0=0$点展开)在计算中会频繁用到,这里建议把每一个都自己手推一遍.
        \begin{enumerate}
            \item $e^z = \sum\dfrac{z^n}{n!},\ (|z|<\infty)$
            \item $\dfrac{1}{1-z}=\sum z^n,\ (|z|<1)$
            \item $\dfrac{1}{1+z}=\sum (-1)^n z^n,\ (|z|<1)$
            \item $\sin{z}=\sum(-1)^n\dfrac{z^{2n+1}}{(2n+1)!},\ (|z|<\infty)$
            \item $\cos{z}=\sum(-1)^n\dfrac{z^{2n}}{(2n)!},\ (|z|<\infty)$
            \item $\ln{(1+z)}=\sum(-1)^n\dfrac{z^{n+1}}{n+1},\ (|z|<1)$
            \item $(1+z)^\alpha=\sum\dfrac{\alpha(\alpha-1)\cdots(\alpha-n+1)}{n!}z^n=\sum\dbinom{\alpha}{n}z^n,\ (|z|<1)$
        \end{enumerate}

        \begin{note}
            这里补充一下普遍二项式的定义.
            \begin{align*}
                \binom{\alpha}{n}=\frac{\alpha(\alpha-1)\cdots(\alpha-n+1)}{n!}
            \end{align*}
        \end{note}

\section{Laurent级数}
    \subsection{Laurent级数的定义}
        \begin{figure}
            \centering
            \includegraphics[width=0.5\textwidth]{AnnularRegion.png}
            \caption{展开为$Laurent$级数的函数解析区域}
            \label{fig:Laurent_series}
        \end{figure}
        \begin{definition}[Laurent级数]\label{def:laurent_series}
            如图\ref{fig:Laurent_series}所示,设$f(z)$在圆环$S$内单值解析,$z$为$S$内任一点,则函数$f(z)$可以展开为幂级数形式.
            \begin{align*}
                f(z)=\sum_{n=-\infty}^{\infty}a_n(z-z_0)^n,\quad a_n=\frac{1}{2\pi i}\oint_{C}\frac{f(\xi)}{(\xi-z_0)^{n+1}}d\xi
            \end{align*}
        \end{definition}
        函数$f(z)$在内圆$C_1$内不解析,但注意$f(z)$在点$z_0$上可能解析也可能不解析.

        $Laurent$级数有两部分,其中正幂项在外圆$C_2$内内闭一致收敛,负幂项在内圆$C_1$外绝对收敛.其中负幂项被称为$Laurent$级数的主要部分,当$Laurent$级数没有
        负幂项时,它就是$Taylor$级数.


    \subsection{Laurent级数的计算}
        废话不多说,直接上例题.
        \begin{example}
            将函数$f(z)=\dfrac{2+3z}{z^2+z^3}$在$z=0$点处写成$Laurent$级数的形式.
        \end{example}
        \begin{solution}
            我们有:
            \begin{align*}
                f(z)&=\frac{1}{z^2}\left(\frac{2+3z}{1+z}\right)=\frac{1}{z^2}\left(3-\frac{1}{1+z}\right)
                =\frac{1}{z^2}\left(3-\sum_{n=0}^{\infty}(-1)^nz^n\right)\\
                &=\frac{1}{z^2}(3-1+z-z^2+z^3-\cdots)=\frac{2}{z^2}+\frac{1}{z}-1+z-z^2+\cdots
            \end{align*}
            该级数在$|z|<1$上收敛.
        \end{solution}

        \begin{example}
            讨论函数$f(z)=1/(4z-z^2)$的$Laurent$展开.
        \end{example}
        \begin{solution}
            该函数有两个奇点$z=0,\,z=4$.所以我们将该函数分别在$z=0$处和$z=\infty$处展开.

            当$|z|<4$时,我们有$|z/4|<1$:
            \begin{align*}
                f(z)=\frac{1}{4z}\left(\frac{1}{1-z/4}\right)=\frac{1}{4z}\sum_{n=0}^{\infty}\left(\frac{z}{4}\right)^n=\sum_{n=0}^{\infty}4^{-n-1}z^{n-1}=\sum_{n=-1}^{\infty}4^{-n-2}z^n.
            \end{align*}

            当$|z|>4$时,我们有$|4/z|>1$(此时是在点$z=\infty$处展开):
            \begin{align*}
                f(z)=-\frac{1}{z^2}\left(\frac{1}{1-4/z}\right)=-\frac{1}{z^2}\sum_{n=0}^{\infty}4^nz^{-n}=-\sum_{-\infty}^{-2}4^{-n-2}z^{n}.
            \end{align*}
        \end{solution}

        从上面这俩例题我们能看到求解$Laurent$级数和使用间接法求解$Taylor$级数十分类似,只是需要注意自变量的取值范围.

    \subsection{奇点的分类}
        定义\ref{def:singular_point}将奇点定义为函数$f$不解析,但在其任意邻域都解析的点,该定义描述的奇点被称为孤立奇点;
        如果在奇点的任一空心领域上都有函数$f$的奇点,这时该奇点就被称为非孤立奇点.孤立奇点根据函数的$Laurent$级数又可以分为可去奇点、极点和本性奇点.
        \begin{enumerate}
            \item 可去奇点\\
                级数展开式不含负幂项.$f(z)=\sum_{n=0}^{\infty}a_n(z-z_0)^n$
            \item 极点\\
                级数展开式含有有限个负幂项,特别的,如果级数展开式含有$m,\ (m\neq \infty)$个负幂项,则称为$m$阶极点.$f(z)=\sum_{n=-m}^{\infty}a_n(z-z_0)^n$
            \item 本性奇点\\
                级数展开式含有无穷个负幂项.$f(z)=\sum_{-\infty}^{\infty}a_n(z-z_0)^n$
        \end{enumerate}

        \begin{note}
            当我们讨论$f(z)$的无穷远点时,我们可以令$z=1/t$,此时点$t=0$的性质就是点$z=\infty$处的性质.
        \end{note}


\section{解析延拓}

    \subsection{解析延拓的定义}

    \subsection{解析延拓的应用}