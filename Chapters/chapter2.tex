\chapter{无穷复数项级数}
\begin{introduction}
    \item 复数项级数
    \item 复幂级数
    \item Taylor级数
    \item Laurent级数
    \item 解析延拓
\end{introduction}

\section{复数项级数}
    \subsection{复数项级数}
    复数项级数和实数项级数完全类似,直接快进到审敛法(

    \subsection{复数项级数的审敛定理}
    这里给出复数项级数的审敛定理.
    \begin{enumerate}
        \item 比值判别法\\
            若$\exists N \subset \mathbbm{N}$,对$\forall n > N$,都有
            $|u_n| < v_n$,并且$\sum_{n = 0}^{\infty}v_n$收敛,则$\sum_{n=0}^{\infty}|u_n|$收敛,即$\sum_{n=0}^{\infty}u_n$绝对收敛.
            若$u_n>v_n>0$,且$\sum_{n=0}^{\infty}v_n$发散,则$\sum_{n=0}^{\infty}|u_n|$发散.
        \item 比值判别法\\
            若存在与$n$无关的常数$\rho$,使得$\left|\dfrac{u_{n+1}}{u_n}\right|<\rho<1$,则级数$\sum_{n=0}^{\infty}u_n$绝对收敛;
            若$\left|\dfrac{u_{n+1}}{u_n}\right|>\rho>1$,则级数发散.
        \item $d'Alembert$判别法\\
            若$\varlimsup_{n\to\infty}|\dfrac{u_{n+1}}{u_n}|<1$,则级数$\sum_{n=0}^{\infty}u_n$绝对收敛;
            若$\varliminf_{n\to\infty}|\dfrac{u_{n+1}}{u_n}|>1$,则级数发散.
        \item $Cauchy$判别法\\
            同$d'Alembert$判别法,只需将$\varlimsup_{n\to\infty}|\dfrac{u_{n+1}}{u_n}|$和$\varliminf_{n\to\infty}|\dfrac{u_{n+1}}{u_n}|$
            均更换为上极限$\varlimsup{{|u_n|}^{1/n}}$.\\
            若$d'Alembert$判别法和$Cauchy$判别法中的两式均为$1$,我们可以使用下面的$Gauss$判别法.
        \item $Gauss$判别法\\
            若$\lim_{n\to\infty}|\dfrac{u_{n+1}}{u_n}|=1$,我们将邻项比值写成如下形式:
            \begin{align*}
                \frac{u_n}{u_{n+1}}=1+\frac{\mu}{n}+O(n^{-\lambda})
            \end{align*}
            这里我们一般通过$Taylor$展开得到.其中$\mu=a+ib,\ \lambda > 1$,若$a>1$,则级数绝对收敛,反之发散.
    \end{enumerate}
    \begin{example}
        判断级数$\sum_{n=1}^{\infty}\dfrac{i^n}{n^\alpha}\,(\alpha>0)$的收敛性和绝对收敛性.
    \end{example}
    \begin{solution}
        我们可以看到,$\sum_{n=1}^{\infty}|\dfrac{i^n}{n^\alpha}|=\sum_{n=1}^{\infty}\dfrac{1}{n^\alpha}$,
        这就是实数项级数中的$p$级数,所以当$\alpha\leq 1$时级数发散,反之绝对收敛.这里我们使用$Gauss$判别法得到相同结论:
        \begin{align*}
            \frac{c_n}{c_{n+1}}=\frac{1/n^\alpha}{1/(n+1)^\alpha}=\left( 1+\frac{1}{n} \right)^\alpha=1+\frac{\alpha}{n}+O\left(\frac{1}{n^2}\right)
        \end{align*}
        当$\alpha \leq 1$时级数发散,反之收敛.
    \end{solution}

\section{复幂级数}
    \subsection{复幂级数的定义和性质}
        \begin{definition}[复幂级数]\label{def:complex_power_series}
            我们将形如$\sum_{k=0}^{\infty}(z-z_0)^k$的复函数项级数称为复幂级数.
        \end{definition}

        \begin{proposition}
            若幂级数$\sum_{k=0}^{\infty}a_k(z - z_0)^k$在点$z_1=\neq z_0$处收敛,则该级数在$|z-z_0|<|z_1-z_0|$处均收敛;
            同样地,若幂级数$\sum_{k=0}^{\infty}b_k(z - z_0)^{(-k)}$在点$z_2=\neq z_0$处收敛,则该级数在$|z-z_0|>|z_2-z_0|$处均收敛
        \end{proposition}

    \subsection{复幂级数的收敛半径计算}
        这部分和实幂级数的计算方法完全一样,但是在计算时要时刻提醒自己算的是复数.
        \begin{enumerate}
            \item 由$Cauchy$判别法,我们可以得到$Cauchy-Hadamard$公式:
                \begin{align}
                    R=\varliminf_{n\to\infty}\left|\frac{1}{c_n}\right|^{1/n}
                \end{align}
            \item 由$d'Alembert$判别法,我们可以得到:
                \begin{align}
                    R=\lim_{n\to\infty}\left|\frac{c_n}{c_{n+1}}\right|
                \end{align}
        \end{enumerate}
        \begin{remark}
            $Cauchy-Hdamard$公式在任意条件下均成立,但$d'Alembert$公式要求有$\lim_{n\to\infty}|c_n/c_{n+1}|$存在.
            求收敛半径时一般先用后者,行不通再使用前者.使用前者时一般要用到重要极限,请牢牢记住$\lim_{f(x)\to\infty}\left(1+\dfrac{1}{f(x)}\right)^{f(x)}=e$.
        \end{remark}

\section{Taylor级数}
    \subsection{Taylor级数的定义}

    \subsection{常见的Taylor级数}

    \subsection{解析函数的零点孤立性和唯一性}

\section{Laurent级数}
    \subsection{Laurent级数的定义}

    \subsection{Laurent级数补充讨论}

    \subsection{常见的Laurent级数}

    \subsection{奇点的分类}


\section{解析延拓}

    \subsection{解析延拓的定义}

    \subsection{解析延拓的应用}