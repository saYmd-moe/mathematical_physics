\chapter{无穷复数项级数}
\begin{introduction}
    \item 复数项级数
    \item 复幂级数
    \item Taylor级数
    \item Laurent级数
    \item 解析延拓
\end{introduction}

\section{复数项级数}
    \subsection{复数项级数}
    复数项级数和实数项级数完全类似,直接快进到审敛法(

    \subsection{复数项级数的审敛定理}
    这里给出复数项级数的审敛定理.
    \begin{enumerate}
        \item 比值判别法\\
                若$\exists N \subset \mathbbm{N}$,对$\forall n > N$,都有
                $|u_n| < v_n$,并且$\sum_{n = 0}^{\infty}v_n$收敛,则$\sum_{n=0}^{\infty}|u_n|$收敛,即$\sum_{n=0}^{\infty}u_n$绝对收敛.
                若$u_n>v_n>0$,且$\sum_{n=0}^{\infty}v_n$发散,则$\sum_{n=0}^{\infty}|u_n|$发散.
        \item 比值判别法\\
                若存在与$n$无关的常数$\rho$,使得$\left|\dfrac{u_{n+1}}{u_n}\right|<\rho<1$,则级数$\sum_{n=0}^{\infty}u_n$绝对收敛;
                若$\left|\dfrac{u_{n+1}}{u_n}\right|>\rho>1$,则级数发散.
        \item $d'Alembert$判别法\\
                若$\varlimsup_{n\to\infty}|\dfrac{u_{n+1}}{u_n}|<1$,则级数$\sum_{n=0}^{\infty}u_n$绝对收敛;
                若$\varliminf_{n\to\infty}|\dfrac{u_{n+1}}{u_n}|>1$,则级数发散.
        \item $Cauchy$判别法\\
                同$d'Alembert$判别法,只需将$\varlimsup_{n\to\infty}|\dfrac{u_{n+1}}{u_n}|$和$\varliminf_{n\to\infty}|\dfrac{u_{n+1}}{u_n}|$
                均更换为上极限$\varlimsup{{|u_n|}^{1/n}}$.\\
                若$d'Alembert$判别法和$Cauchy$判别法中的两式均为$1$,我们可以使用下面的$Gauss$判别法.
        \item $Gauss$判别法\\
                若$\lim_{n\to\infty}|\dfrac{u_{n+1}}{u_n}|=1$,我们将邻项比值写成如下形式:
                \begin{align*}
                    \frac{u_n}{u_{n+1}}=1+\frac{\mu}{n}+O(n^{-\lambda})
                \end{align*}
                这里我们一般通过$Taylor$展开得到.其中$\mu=a+ib,\ \lambda > 1$,若$a>1$,则级数绝对收敛,反之发散.
    \end{enumerate}
    \begin{example}
        判断级数$\sum_{n=1}^{\infty}\dfrac{i^n}{n^\alpha}\,(\alpha>0)$的收敛性和绝对收敛性.
    \end{example}
    \begin{solution}
        我们可以看到,$\sum_{n=1}^{\infty}|\dfrac{i^n}{n^\alpha}|=\sum_{n=1}^{\infty}\dfrac{1}{n^\alpha}$,
        这就是实数项级数中的$p$级数,所以当$\alpha\leq 1$时级数发散,反之绝对收敛.这里我们使用$Gauss$判别法得到相同结论:
        \begin{align*}
            \frac{c_n}{c_{n+1}}=\frac{1/n^\alpha}{1/(n+1)^\alpha}=\left( 1+\frac{1}{n} \right)^\alpha=1+\frac{\alpha}{n}+O\left(\frac{1}{n^2}\right)
        \end{align*}
        当$\alpha \leq 1$时级数发散,反之收敛.
    \end{solution}

\section{复幂级数}
    \subsection{复幂级数的定义}

    \subsection{复幂级数的敛散性判断}

    \subsection{复幂级数的收敛半径计算}

    \subsection{复幂级数和函数的解析性及其性质}

\section{Taylor级数}
    \subsection{Taylor级数的定义}

    \subsection{常见的Taylor级数}

    \subsection{解析函数的零点孤立性和唯一性}

\section{Laurent级数}
    \subsection{Laurent级数的定义}

    \subsection{Laurent级数补充讨论}

    \subsection{常见的Laurent级数}

    \subsection{奇点的分类}


\section{解析延拓}

    \subsection{解析延拓的定义}

    \subsection{解析延拓的应用}